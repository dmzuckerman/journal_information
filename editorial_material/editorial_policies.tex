\documentclass{article}
\usepackage{graphicx}
\usepackage{hyperref}

\begin{document}

\title{Editorial Policies of LiveCoMS}
\author{Editorial Board of LiveCoMS}

\maketitle

\begin{abstract}
The current draft list of editorial board policies for the Living
Journal of Computational Molecular Science (LiveCoMS). These policies
are not yet active, as they are still in draft.
\end{abstract}

\section{Policies on Submissions to LiveCoMS}

\subsection{Policy on Licensing}

LiveCoMS does not ask for or request any copyright transfer.

However, in order to submit to LiveCoMS, authors must provide, at
minimum, a license for LiveCoMS to publish the article and distribute
it free of charge. We recommend that the authors release the article
under an open source license such as \href{https://creativecommons.org/licenses/by/4.0/}{Creative Commons
  Attribution} (also
known as CC-BY) releasing the document for anyone to copy and
redistribute the material in any medium or format, and remix,
transform, and build upon the material for any purpose, even
commercially.  Making it available to all of course makes it possible
for LiveCoMS to publish it, and for the community to edit and
contribute.  We highly recommend the CC-BY license in order to ensure
your work can reach and help the broadest audience possible, and
suggest that when considering the appropriate license you \href{http://openaccess.ox.ac.uk/2013/06/13/cc-by-what-does-it-mean-for-scholarly-articles-3/}{read this analysis}.

Other more restrictive licenses may be permissible as long as LiveCoMS
has the permission to publish and excerpt from the document; this
might be required if someone other than the authors has some rights to
the material (for example, if it was previously published in some
other journal). Generally, we only allow a more restrictive license in
these cases, but are happy to discuss. Please ask the managing editors
if you require some other licensing regime.

For employees of the U.S. Government, their work products are under
public domain, and thus CC-BY is not appropriate. We recommend the
license language: ``As a work of the United States Government, this
package is in the public domain within the United
States. Additionally, [Agency Name] waives copyright and related
rights in the work worldwide through the CC0 1.0 Universal Public
Domain Dedication (which can be found at
https://creativecommons.org/publicdomain/zero/1.0/).''\href{https://theunitedstates.io/licensing/}{See the analysis of this language here}.

\subsection{Policies on Author Order}
Authors should, to the extent possible, determine the author order
among themselves.  Each work must have a section describing the actual
contributions of authors (and of those acknowledged) to provide
clarity, and journal templates include such a section.  
%MRS: still need to add that section.  We should also likely provide
%more explicit instructions here as to how finely to divide the contributions.

However, we acknowledge that institutions evaluating for merit, promotion, tenure
or other cases may not read this level of detail, so traditional
notions of author order (first author, corresponding author, etc.) may
still be relevant and the authors will need to coordinate who should
occupy these positions.


\subsection{Policy on Abandoned Documents}

By the submission of their paper, document, or materials, all the
authors consent that if they no longer are willing or able to maintain
their work, LiveCoMS may assign another individual or individuals to
do so, with appropriate modifications to the authors list. Authors
would be given written notice (by e-mail and formal letter) if this
were to happen and would have a period of six months to respond and/or
designate a successor before LiveCoMS would do so for them. Typically,
authors are implicitly giving a right to do this via licensing under
\href{https://creativecommons.org/licenses/by/4.0/}{Creative Commons -
  Attribution} or
similar licenses which give others the right to create derivative
works (potentially allowing others to ``resurrect'' a document which
has been abandoned); however, we expect all authors to explicitly
consent to this policy to avoid any confusion. Ordinarily, we expect
this policy will be relevant only in unusual or extreme cases where an
author or authors dies or leaves the field; in most other cases
authors will presumably be available to designate their own successors
or succession plan if a work is valuable to the field and will
continue to need maintenance and the original author(s) are no longer
willing or able to do so. However, we want to plan for the possibility
of unusual events, hence our need for this policy.

Thus, for these reasons, authors submitting to LiveCoMS are agreeing
that others may take over authorship of their article (with
appropriate acknowledgment/recognition of the original authors) if
they have abandoned their article as described above, and that
LiveCoMS may accept revised versions of papers which have amended
authorship in such circumstances.

\section{Policies About the Editorial Boards}

\subsection{Policies on Decisions}
Issues regarding governance structure, editorial board membership, leadership, and financial
model must be decided by a 2/3 majority vote of the voting members of the editorial
board.

Other issues, such as editorial policy, licensing policy, can be
decided by a majority vote.

Votes will be are tallied based on the fraction of the editorial board
responding within one week written notice via e-mail. In exceptional
circumstances advance notice may be given of a rush decision, where
the board may be given one week (or more) notice that their votes will
be needed within a shorter window, such as a 12 hour window.

\end{document}

% LocalWords:  LiveCoMS remix contactable CC
