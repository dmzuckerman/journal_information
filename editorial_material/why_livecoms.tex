%%%%%%%%%%%%%%%%%%%%%%%%%%%%%%%%%%%%%%%%%%%%%%%%%%%%%%%%%%%%
%%% LIVECOMS ARTICLE TEMPLATE
%%% ADAPTED FROM ELIFE ARTICLE TEMPLATE (8/10/2017)
%%%%%%%%%%%%%%%%%%%%%%%%%%%%%%%%%%%%%%%%%%%%%%%%%%%%%%%%%%%%
%%% PREAMBLE 
\documentclass[9pt]{livecoms}
% Use the onehalfspacing option for 1.5 line spacing
% Use the doublespacing option for 2.0 line spacing
% Please note that these options may affect formatting.

\usepackage{lipsum} % Required to insert dummy text
\usepackage[version=4]{mhchem}
\usepackage{siunitx}
\DeclareSIUnit\Molar{M}

%%%%%%%%%%%%%%%%%%%%%%%%%%%%%%%%%%%%%%%%%%%%%%%%%%%%%%%%%%%%
%%% ARTICLE SETUP
%%%%%%%%%%%%%%%%%%%%%%%%%%%%%%%%%%%%%%%%%%%%%%%%%%%%%%%%%%%%
\title{Why we need the Living Journal of Computational Molecular Sciences}

\author[1*]{David L. Mobley}
\author[2*]{Michael R. Shirts}
\author[3*]{Daniel M. Zuckerman}
\affil[1]{Department of Pharmaceutical Sciences and Chemistry, University of California, Irvine}
\affil[2]{Department of Chemical and Biological Engineering, University of Colorado Boulder}
\affil[2]{Department of Biomedical Engineering, Oregon Health Sciences University}

\corr{dmobley@mobleylab.org}{DLM}
\corr{michael.shirts@colorado.edu}{MRS}
\corr{zuckermd@ohsu.edu}{DMZ}


%%%%%%%%%%%%%%%%%%%%%%%%%%%%%%%%%%%%%%%%%%%%%%%%%%%%%%%%%%%%
%%% ARTICLE START
%%%%%%%%%%%%%%%%%%%%%%%%%%%%%%%%%%%%%%%%%%%%%%%%%%%%%%%%%%%%

\begin{document}

\maketitle

\begin{abstract}
The Living Journal of Computational Molecular Sciences (LiveCoMS) seeks to bring publishing of educational, review, and best practices materials into the Internet age, allowing authors to publish their work but then continue its development.
We provide a venue where authors can provide living documents that are updated on an ongoing basis as websites or Wikipedia articles could be, but which still have clear authorship and provide a mechanism for authors to get publication credit for their work.
To some extent, LiveCoMS is a publishing experiment, but one which we hope will help advance publishing towards the future we want.
Readers should know where to find authoritative information on a given topic and be confident that it is being kept up to date; our model allows for that. 
Additionally, authors who invest a great deal of time and effort in creating valuable materials to advance the field should receive credit for it without having to re-write the same material again to re-publish it in a different journal as the field moves on. 
We believe the model presented by LiveCoMS addresses these issues.
\end{abstract}


\section{LiveCoMS focuses providing living documents to advance the field}

LiveCoMS seeks to improve the availability, quality, and timeliness of educational, review, and best practices papers in the computational molecular sciences (CoMS).
Research in CoMS requires significant training and expertise in diverse fields and a breadth of background knowledge, requiring high quality training materials covering a broad range of topics.
At the same time, our science advances quickly, and it is difficult for these materials to keep pace. 
Our training tends to rely primarily on textbooks and review papers, both of which are updated only sporadically, partly because easier-to-update documents like Wikipedia and websites don't provide a mechanism for the academic credit we often rely on [ref section below].
LiveCoMS seeks to change this, providing high quality but rapidly updatable materials covering important areas in CoMS.

Currently, both textbooks and review papers are difficult to update or maintain.
For example, Leach's Molecular Modelling [ref, formatting] remains a standard and valuable text in the field but was last updated in 2001.
Allen and Tildesley's text [ref] is also standard and went from 1989 to 2017 between updates. 
While review papers tend to be more timely, these also go out of date quickly, and are often limited by space. 
Typical journal copyright transfer policies mean that a journal takes ownership of the content of a review when it is published and does not allow creation of derivative works, so authors have no mechanism to update and re-publish their work.
Thus, review papers typically provide a snapshot of the literature and the authors' current perspective when the article was published, but they become less valuable as time passes and the field moves on.
This is part of the reason why authors who stay in a given field may publish new review papers of their area every few years. 

LiveCoMS seeks to solve three challenges simultaneously: We need (1) timely and high quality documents to advance our science which are (2) updatable, with (3) incentives provided so authors receive appropriate credit for their contributions.
We believe our model will allow for exactly this. 

\section{Preprints, perpetual/living reviews, and overlay journals provide context for LiveCoMS}

While much of traditional academic publishing has yet to be dramatically reshaped by the Internet age, the present day surge of interest in preprints and preprint servers provides a glimpse of the path forward.
With preprints, authors can post their articles in advance formal publication, gaining readers, drawing community attention, getting citations, and even receiving feedback and constructive criticism that authors use make these articles stronger before their formal publication.
Thus, preprints provide a glimpse of a path forward, as they are updatable documents which are current and have real value.
Posting of preprints is also typically free, lowering the entry barrier. 
However, in much of computational molecular science (some areas of physics being a potential exception) preprints do not yet provide the same degree of academic credit as a peer-reviewed publication. 

Another glimpse is provided by the concept of ``perpetual'' or ``living'' reviews [ref Zuckerman/Mobley and other references therein]; these begin as traditional review papers but the authors keep them current by releasing new versions regularly which incorporate new developments and/or community feedback.
These provide the potential that an author who stays in a given field could maintain an authoritative, current document which always reflect the state of the field and the author(s)' views. 
One of the present authors (DLM) is currently experimenting with just such a review [ref], and it has already expanded significantly to incorporate new developments, even gaining an additional author.

Still, neither of these advancements provide a mechanism for publication credit; this type of credit can be provided by the concept of ``overlay journals''.
Overlay journals are journals which don't formally host articles themselves, but which provide peer reviews of specific versions of preprints, then, based on these reviews, publishes articles by link to them and giving them status as formal publications.
Examples of such publications include Discrete Analysis [ref website: http://discreteanalysisjournal.com/ and blog explanation: https://gowers.wordpress.com/2015/09/10/discrete-analysis-an-arxiv-overlay-journal/], Quantum [http://quantum-journal.org/instructions/authors/], and  [add others].
This allows fairly formal publication credit; for example, both are indexed by Google Scholar [confirm], and Quantum is working towards inclusion in Web of Science and other indices. 
At the same time, the overlay journal approach also allows costs to remain extremely low; these journals typically charge a nominal fee (\$10 to \$200 or so) to cover operating costs and the costs of managing peer review, but this is quite affordable and authors can in some cases draw on institutional open access incentives to cover the costs. 

\section{LiveCoMS combines the concepts of preprints, perpetual reviews, and overlay journals to achieve our goals}

LiveCoMS achieves its goals by providing an overlay journal allowing versioned updates.
Authors post documents to their preprint server of choice, gaining the benefits of traditional preprints such as early exposure, community feedback, and so on. 
LiveCoMS then arranges for peer reviews of these preprints, and, after review/revision, if the articles are suitable for publication, it formally publishes them by linking to the specific version which was reviewed and [do we give them DOIs or anything else?].
This provides formal publication credit, yet it still allows straightforward versioned updates; the authors are free to post new versions of their ``preprints'' at any time, and can also update the version linked to by LiveCoMS when needed, as discussed below.

To help with handling of the updating process and community feedback, we recommend the ``paper writing as code development'' model [link] to formalize community feedback and provide better mechanisms for outside contributions. 

Revision of works published in LiveCoMS can occur at two levels. 
First, for minor revisions, LiveCoMS publications can link to the preprint server or other repository where ALL versions of a paper will be placed, so readers visiting the published version of the article will be able to click through to the most up-to-date version.
Second, significant revisions are expected to be made on a fairly regularly basis (perhaps yearly to every couple of years) and these can be published again in LiveCoMS, allowing authors to obtain additional publication credit for work they have done updating and maintaining these. 

%MRS: should this section below go earlier?
\section{LiveCoMS focuses on documents which benefit from or require this model}

LiveCoMS focuses on publishing content which needs regular updating and which can be of real long-term benefit from the community.
We focus especially on:
\begin{enumerate}
\item Perpetual reviews, review papers which will need updating on an ongoing basis
\item Best practices documents to help ensure these are followed in computation, providing valuable training which needs updating as science advances
\item Tutorial documents which educate CoMS scientists on how to perform specific, important tasks which need maintenance as codes/tools change
\item Simulation/software package comparisons and validation, which likewise need updating as new versions are released
\end{enumerate}

\section{Conclusions}

LiveCoMS is an experiment we very much believe in, one we think can provide a source for a wealth of valuable educational, training, and review material, while providing researchers with the incentives they need so the effort they invest will pay off.
We hope you agree and will consider submitting your work [link] to LiveCoMS.


\section{Acknowledgments}

We thank the University of Colorado Boulder for initial financial support for LiveCoMS. 
DLM thanks Michael K. Gilson (UCSD) for helping with and tolerating his initial experiments on a perpetual review which was first published in Annual Reviews [ref] and then continues to be updated on GitHub [ref].

%\nocite{*} % This command displays all refs in the bib file
\bibliography{why_livecoms}

\end{document}
