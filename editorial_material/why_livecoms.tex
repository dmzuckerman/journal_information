%%%%%%%%%%%%%%%%%%%%%%%%%%%%%%%%%%%%%%%%%%%%%%%%%%%%%%%%%%%%
%%% LIVECOMS ARTICLE TEMPLATE
%%% ADAPTED FROM ELIFE ARTICLE TEMPLATE (8/10/2017)
%%%%%%%%%%%%%%%%%%%%%%%%%%%%%%%%%%%%%%%%%%%%%%%%%%%%%%%%%%%%
%%% PREAMBLE 
\documentclass[9pt,lineno]{livecoms}
% Use the onehalfspacing option for 1.5 line spacing
% Use the doublespacing option for 2.0 line spacing
% Please note that these options may affect formatting.

\usepackage{lipsum} % Required to insert dummy text
\usepackage[version=4]{mhchem}
\usepackage{siunitx}
\DeclareSIUnit\Molar{M}

%%%%%%%%%%%%%%%%%%%%%%%%%%%%%%%%%%%%%%%%%%%%%%%%%%%%%%%%%%%%
%%% ARTICLE SETUP
%%%%%%%%%%%%%%%%%%%%%%%%%%%%%%%%%%%%%%%%%%%%%%%%%%%%%%%%%%%%
\title{Why we need the Living Journal of Computational Molecular Sciences}

\author[1*]{David L. Mobley}
\author[2*]{Michael R. Shirts}
\author[3*]{Daniel M. Zuckerman}
\affil[1]{Department of Pharmaceutical Sciences and Chemistry, University of California, Irvine}
\affil[2]{Department of Chemical Engineering, University of Colorado, Boulder}
\affil[2]{Department of Biomedical Engineering, Oregon Health Sciences University}

\corr{dmobley@mobleylab.org}{DLM}
\corr{michael.shirts@colorado.edu}{MRS}
\corr{zuckermd@ohsu.edu}{DMZ}


%%%%%%%%%%%%%%%%%%%%%%%%%%%%%%%%%%%%%%%%%%%%%%%%%%%%%%%%%%%%
%%% ARTICLE START
%%%%%%%%%%%%%%%%%%%%%%%%%%%%%%%%%%%%%%%%%%%%%%%%%%%%%%%%%%%%

\begin{document}

\maketitle

\begin{abstract}
The Living Journal of Computational Molecular Sciences (LiveCoMS) seeks to bring publishing of educational, review, and best practices materials into the Internet age, allowing authors to publish their work but then continue its development.
We provide a venue where authors can provide living documents that are updated on an ongoing basis as websites or Wikipedia articles could be, but which still have clear authorship and provide a mechanism for authors to get publication credit for their work.
To some extent, LiveCoMS is a publishing experiment, but one which we hope will help advance publishing towards the future we want.
Readers should know where to find authoritative information on a given topic and be confident that it is being kept up to date; our model allows for that. 
And authors who invest a great deal of time and effort in creating valuable materials to advance the field should receive credit for it without having to re-write the same material again to re-publish it in a different journal as the field moves on. 
We believe LiveCoMS' model address both of these issues.
\end{abstract}


\section{LiveCoMS focuses providing living documents to advance the field}

LiveCoMS seeks to improve the availability, quality, and timeliness of educational, review, and best practices papers in the computational molecular sciences (CoMS).
Research in CoMS requires significant training and expertise in diverse fields and a breadth of background knowledge, requiring high quality training materials covering a broad range of topics.
At the same time, our science advances quickly, and it is difficult for these materials to keep pace. 
Our training tends to rely primarily on textbooks and review papers, both of which are updated only sporadically, partly because easier-to-update documents like Wikipedia and websites don't provide a mechanism for the academic credit we often rely on [ref section below].
LiveCoMS seeks to change this, providing high quality but rapidly updatable materials covering important areas in CoMS.

Currently, both textbooks and review papers are difficult to update or maintain.
For example, Leach's Molecular Modelling [ref, formatting] remains a standard and valuable text in the field but was last updated in 2001.
Allen and Tildesley's text [ref] is also standard and went from 1989 to 2017 between updates. 
While review papers tend to be more timely, these also go out of date. 
Typical journal copyright transfer policies mean that a journal takes ownership of the content of a review when it is published and does not allow creation of derivative works, so authors have no mechanism to update and re-publish their work.
Thus, review papers typically provide a snapshot of the literature and the authors' current perspective when the article was published, but they become less valuable as time passes and the field moves on.
This is part of the reason why authors who stay in a given field may publish new review papers of their area every several or few years. 

LiveCoMS seeks to solve three challenges simultaneously: We need (1) timely and high quality documents to advance our science which are (2) updatable, with (3) incentives provided so authors receive appropriate credit for their contributions.
We believe our model will allow for exactly this. 

\section{Preprints, perpetual/living reviews, and overlay journals provide context for LiveCoMS}

   - Preprints provide a glimpse of a path forward; authors post, get readers, draw community attention, receive feedback, and make updates accordingly
   - Another glimpse: perpetual/living reviews as an alternative to traditional reviews
   - Overlay journals: Providing a way to "publish" preprints
- LiveCoMS achieves its goals by providing an overlay journal allowing versioned updates
   - Authors post as preprint, gaining benefits of traditional preprints
   - We also recommend the "paper writing as code development" model (link) to formalize community feedback and mechanisms for outside contributions
   - LiveCoMS arranges for peer reviews of these preprints, then publishes them by linking to them
   - Provides formal publication credit, yet still allows for straightforward versioned updates
      - Two levels of updates: Minor (not requiring additional peer review and not counting as a new publication) and major (requiring additional peer review and resulting in a new publication)
- LiveCoMS content goals focus on material requiring this model
   - Reviews
   - Best practices
   - Tutorials
   - Simulation package comparisons/validation


\section{Results (Level 1 heading)}

\lipsum[2-3]

\begin{table}[bt]
\caption{\label{tab:example}Automobile Land Speed Records (GR 5-10).}
% Use ``S'' column identifier to align on decimal point 
\begin{tabular}{S l l l r}
\toprule
{Speed (mph)} & Driver          & Car                        & Engine    & Date     \\
\midrule
407.447     & Craig Breedlove & Spirit of America          & GE J47    & 8/5/63   \\
413.199     & Tom Green       & Wingfoot Express           & WE J46    & 10/2/64  \\
434.22      & Art Arfons      & Green Monster              & GE J79    & 10/5/64  \\
468.719     & Craig Breedlove & Spirit of America          & GE J79    & 10/13/64 \\
526.277     & Craig Breedlove & Spirit of America          & GE J79    & 10/15/65 \\
536.712     & Art Arfons      & Green Monster              & GE J79    & 10/27/65 \\
555.127     & Craig Breedlove & Spirit of America, Sonic 1 & GE J79    & 11/2/65  \\
576.553     & Art Arfons      & Green Monster              & GE J79    & 11/7/65  \\
600.601     & Craig Breedlove & Spirit of America, Sonic 1 & GE J79    & 11/15/65 \\
622.407     & Gary Gabelich   & Blue Flame                 & Rocket    & 10/23/70 \\
633.468     & Richard Noble   & Thrust 2                   & RR RG 146 & 10/4/83  \\
763.035     & Andy Green      & Thrust SSC                 & RR Spey   & 10/15/97\\
\bottomrule
\end{tabular}

\medskip 
Source: \url{https://www.sedl.org/afterschool/toolkits/science/pdf/ast_sci_data_tables_sample.pdf}

\tabledata{This is a description of a data source.}

\end{table}

\subsection{Level 2 Heading}

\lipsum[3]

\subsubsection{Level 3 Heading}

\lipsum[5]

\paragraph{Level 4 Heading}
\lipsum[7]

\section{Discussion}

\lipsum[9]

\section{Methods and Materials}

Guidelines can be included for standard research article sections, such as this one. 

\lipsum[3]

\section{Some \LaTeX{} Examples}
\label{sec:examples}

Use section and subsection commands to organize your document. \LaTeX{} handles all the formatting and numbering automatically. Use ref and label commands for cross-references.

\subsection{Figures and Tables}

Use the table and tabular commands for basic tables --- see \TABLE{example}, for example. 

You can upload a figure (JPEG, PNG or PDF) using the project menu. To include it in your document, use the \verb|\includegraphics| command as in the code for \FIG{view}. 

For a half-width figure or table with text wrapping around it, use 

\begin{verbatim}
\begin{wrapfigure}{l}{.46\textwidth}
 % \includegraphics[width=\hsize]{...}
  \caption{...}\label{...}
\end{wrapfigure}
\end{verbatim}
%
as in \FIG{halfwidth}. For tables:

\begin{verbatim}
\begin{wraptable}{l}{.46\textwidth}{
  \begin{tabular}{...}
  ...
  \end{tabular}}
  \caption{...}\label{...}
\end{wraptable}
\end{verbatim}

Be careful with these, though, as they may behave strangely near page boundaries, sectional headings, or in the neighbourhood of lists or too many floats.

If you use the following prefixes for your \verb|\label|:
%
\begin{description}
\item[Figures] \texttt{fig:}, e.g.~\verb|\label{fig:view}|
\item[Tables] \texttt{tab:}, e.g.~\verb|\label{tab:example}|
\item[Equations] \texttt{eq:}, e.g.~\verb|\label{eq:CLT}|
\item[Boxes] \texttt{box:}, e.g.~\verb|\label{box:simple}|
\end{description}
%
you can then use the convenience commands \verb|\FIG{view}|, \verb|\TABLE{example}|, \verb|\EQ{CLT}| and \verb|\BOX{simple}| \emph{without} the label prefixes, to generate cross-references \FIG{view}, \TABLE{example}, \EQ{CLT} and \BOX{simple}. Alternatively, use \verb|\autoref| with the full label, e.g.~\autoref{first:app} (although this may not work correctly for figures and tables in the appendices or boxes nor supplements at present).

Really wide figures or tables, that take up the entire page, including the gutter space: use \verb|\begin{fullwidth}...\end{fullwidth}| as in \FIG{fullwidth}. And sometimes you may want to use feature boxes like \BOX{simple}.

\begin{wrapfigure}{l}{.46\textwidth}
%\includegraphics[width=\hsize]{frog}
\caption{A half-columnwidth image using wrapfigure, to be used sparingly. Note that using a wrapfigure before a sectional heading, near other floats or page boundaries is not recommended, as it may cause interesting layout issues. Use the optional argument to wrapfigure to control how many lines of text should be set half-width alongside it.}
\label{fig:halfwidth}
\end{wrapfigure}

Some filler text to sit alongside the half-width figure. \lipsum[1-2]

\begin{figure}
\begin{fullwidth}
%\includegraphics[width=0.95\linewidth]{elife-18156-fig2}
\caption{A very wide figure that takes up the entire page, including the gutter space.}
\label{fig:fullwidth}
%\figsupp{There is no limit on the number of Figure Supplements for any one primary figure. Each figure supplement should be clearly labelled, Figure 1--Figure Supplement 1, Figure 1--Figure Supplement 2, Figure 2--Figure Supplement 1 and so on, and have a short title (and optional legend). Figure Supplements should be referred to in the legend of the associated primary figure, and should also be listed at the end of the article text file.}{\includegraphics[width=6cm]{frog}}
\end{fullwidth}
\end{figure}

\subsection{Citations}

LaTeX formats citations and references automatically using the bibliography records in your .bib file, which you can edit via the project menu. Use the \verb|\cite| command for an inline citation, like \cite{Aivazian917}, and the \verb|\citep| command for a citation in parentheses \citep{Aivazian917}. The LaTeX template uses a slightly-modified Vancouver bibliography style. If your manuscript is accepted, the eLife production team will re-format the references into the final published form. \emph{It is not necessary to attempt to format the reference list yourself to mirror the final published form.}

\begin{featurebox}
\caption{This is an example feature box}
\label{box:simple}
This is a feature box. It floats!
\medskip

\includegraphics[width=5cm]{example-image}
\featurefig{`Figure' and `table' captions in feature boxes should be entered with \texttt{\textbackslash featurefig} and \texttt{\textbackslash featuretable}. They're not really floats.}

\lipsum[1]
\end{featurebox}

\subsection{Mathematics}

\LaTeX{} is great at typesetting mathematics. Let $X_1, X_2, \ldots, X_n$ be a sequence of independent and identically distributed random variables with $\text{E}[X_i] = \mu$ and $\text{Var}[X_i] = \sigma^2 < \infty$, and let
\begin{equation}
\label{eq:CLT}
S_n = \frac{X_1 + X_2 + \cdots + X_n}{n}
      = \frac{1}{n}\sum_{i}^{n} X_i
\end{equation}
denote their mean. Then as $n$ approaches infinity, the random variables $\sqrt{n}(S_n - \mu)$ converge in distribution to a normal $\mathcal{N}(0, \sigma^2)$.

\lipsum[3]

\begin{figure}
%\includegraphics[width=\linewidth]{elife-13214-fig7}
\caption{A text-width example.}
\label{fig:view}
%% If the optional argument in the square brackets is ``none'', then the caption *will not appear in the main figure at all* and only the full caption will appear under the supplementary figure at the end of the manuscript.
%\figsupp[Shorter caption for main text.]{This is a supplementary figure's full caption, which will be used at the end of the manuscript.}{\includegraphics[width=6cm]{frog}}
%\figsupp{This is another supplementary figure.}{\includegraphics[width=6cm]{frog}}
\figdata{This is a description of a data source.}
\figdata{This is another description of a data source.}
\end{figure}

\subsection{Other Chemistry Niceties}

You can use commands from the \texttt{mhchem} and \texttt{siunitx} packages. For example: \ce{C32H64NO7S}; \SI{5}{\micro\metre}; \SI{30}{\degreeCelsius}; \SI{5e-17}{\Molar}

\subsection{Lists}

You can make lists with automatic numbering \dots

\begin{enumerate}
\item Like this,
\item and like this.
\end{enumerate}
\dots or bullet points \dots
\begin{itemize} 
\item Like this,
\item and like this.
\end{itemize}
\dots or with words and descriptions \dots
\begin{description}
\item[Word] Definition
\item[Concept] Explanation
\item[Idea] Text
\end{description}

Some filler text, because empty templates look really poorly. \lipsum[1]


\section{Acknowledgments}

Additional information can be given in the template, such as to not include funder information in the acknowledgments section.

\nocite{*} % This command displays all refs in the bib file
\bibliography{elife-sample}

%%%%%%%%%%%%%%%%%%%%%%%%%%%%%%%%%%%%%%%%%%%%%%%%%%%%%%%%%%%%
%%% APPENDICES
%%%%%%%%%%%%%%%%%%%%%%%%%%%%%%%%%%%%%%%%%%%%%%%%%%%%%%%%%%%%

\appendix
\begin{appendixbox}
\label{first:app}
\section{Firstly}
\lipsum[1]

%% Sadly, we can't use floats in the appendix boxes. So they don't ``float'', but use \captionof{figure}{...} and \captionof{table}{...} to get them properly caption.
\begin{center}
%\includegraphics[width=\linewidth,height=7cm]{frog}
\captionof{figure}{This is a figure in the appendix}
\end{center}

\section{Secondly}

\lipsum[5-8]

\begin{center}
%\includegraphics[width=\linewidth,height=7cm]{frog}
\captionof{figure}{This is a figure in the appendix}
\end{center}

\end{appendixbox}

\begin{appendixbox}
%\includegraphics[width=\linewidth,height=7cm]{frog}
\captionof{figure}{This is a figure in the appendix}
\end{appendixbox}
\end{document}
